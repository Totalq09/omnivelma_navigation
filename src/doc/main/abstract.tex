
\begin{center}
\begin{large}
\textbf{Streszczenie}
\end{large}
\end{center}

\noindent{\textbf{Tytuł: }} Unikanie kolizji człowiek-robot z wykorzystaniem czujnika Kinect \\
\noindent{\textbf{Słowa kluczowe: }} ROS, Gazebo, symulacja, nawigacja, kolizja, detekcja, człowiek \\

Celem pracy inżynierskiej było przygotowanie systemu pozwalającego na unikanie kolizji człowiek-robot. W celu realizacji zadania posłużono się symulatorem platformy dookólnej.\\
\indent W niniejszej pracy omówione zostają problemy specyfiki działania robota w środowisku człowieka, wymagania jakie powinien spełniać budowany system oraz zaprezentowany zostaje jego projekt i implementacja.\\ 
\indent Praca wymagała zrealizowania systemu nawigacyjnego, który na podstawie danych z receptorów i odometrii robota, jest wstanie autonomicznie kierować robotem. Ważnym elementem systemu jest zdolność detekcji człowieka. W tym celu wykorzystano skaner laserowy (LIDAR), sensor RGB-D Kinect oraz odpowiednie algorytmy przetwarzania obrazu, dzięki którym możliwe było uwzględnienie strefy osobistej człowieka. \\
\indent System zbudowano z wykorzystaniem programowej struktury ramowej ROS, bibliotek OpenCV i Dlib oraz symulatora Gazebo.

\newpage

\begin{center}
\begin{large}
\textbf{Abstract}
\end{large}
\end{center}

\noindent{\textbf{Title: }} Avoiding human-robot collision using Kinect sensor\\
\noindent{\textbf{Keywords: }} ROS, Gazebo, simulation, navigation, collision, detection, human-aware \\

The aim of thesis was to design human-robot collision avoidance system. In order to accomplish this goal, the omnidirectional mobile platform was used.\\
\indent This paper discusses issues concerning environment that contains people, specification of working in such environment, requirements which need to be met. Afterwards, project and implementation of particular system are presented.\\
\indent Thesis required implementation of navigation system that allows  autonomious navigation, based on received sensor's data and robot's odometry. Crucial part of system is ability to detect human and take human's personal space into consideration. LIDAR sensor, Kinect RGB-D sensor and proper image processing algorithms were utilized for that purpose. \\
\indent System was implemented based on ROS framework, OpenCV and Dlib libraries and Gazebo simulator. 



\newpage